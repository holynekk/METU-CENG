\documentclass[12pt]{article}
\usepackage[utf8]{inputenc}
\usepackage[dvips]{graphicx}
\usepackage{epsfig}
\usepackage{fancybox}
\usepackage{verbatim}
\usepackage{array}
\usepackage{latexsym}
\usepackage{alltt}
\usepackage{float}
\usepackage{amsmath}
\usepackage{hyperref}
\usepackage{listings}
\usepackage{color}
\usepackage[hmargin=3cm,vmargin=5.0cm]{geometry}
\topmargin=-1.8cm
\addtolength{\textheight}{6.5cm}
\addtolength{\textwidth}{2.0cm}
\setlength{\oddsidemargin}{0.0cm}
\setlength{\evensidemargin}{0.0cm}

\newcommand{\HRule}{\rule{\linewidth}{1mm}}
\newcommand{\kutu}[2]{\framebox[#1mm]{\rule[-2mm]{0mm}{#2mm}}}
\newcommand{\gap}{ \\[1mm] }

\newcommand{\Q}{\raisebox{1.7pt}{$\scriptstyle\bigcirc$}}

\lstset{
    %backgroundcolor=\color{lbcolor},
    tabsize=2,
    language=C++,
    basicstyle=\footnotesize,
    numberstyle=\footnotesize,
    aboveskip={0.0\baselineskip},
    belowskip={0.0\baselineskip},
    columns=fixed,
    showstringspaces=false,
    breaklines=true,
    prebreak=\raisebox{0ex}[0ex][0ex]{\ensuremath{\hookleftarrow}},
    %frame=single,
    showtabs=false,
    showspaces=false,
    showstringspaces=false,
    identifierstyle=\ttfamily,
    keywordstyle=\color[rgb]{0,0,1},
    commentstyle=\color[rgb]{0.133,0.545,0.133},
    stringstyle=\color[rgb]{0.627,0.126,0.941},
}


\begin{document}



\noindent
\HRule \\[3mm]
\small
\begin{tabular}[b]{lp{3.8cm}r}
{} Middle East Technical University &  &
{} Department of Computer Engineering \\
\end{tabular} \\
\begin{center}

                 \LARGE \textbf{CENG 223} \\[4mm]
                 \Large Discrete Computational Structures \\[4mm]
                \normalsize Fall '2020-2021 \\
                    \Large Homework 3 \\
                \normalsize Student Name and Surname:  Mert Kaan YILMAZ\\
                \normalsize Student Number:  2381093\\
\end{center}
\HRule


\section*{Question 1}
For the Fermat's Little Theorem, $a^{p-1}\equiv 1(mod\:p)$, if p is prime and a cannot be divided by p. Also for every integer a, $a^p\equiv a(mod\:p)$.\\
Then,\\
$(2^{22}+4^{44}+6^{66}+8^{80}+10^{110})(mod\:11)$\\
$\equiv\:(2^2\cdot(2^2)^{10}+4^4\cdot(4^4)^{10}+6^6\cdot(6^6)^{10}+(8^8)^{10}+(10^{11})^{10})(mod\:11)$\\
$\equiv\:(4\cdot1+16\cdot16\cdot1+36\cdot36\cdot36\cdot1+1+1)(mod\:11)$\\
$\equiv\:(4+5\cdot5+3\cdot3\cdot3+1+1)(mod\:11)$\\
$\equiv\:58(mod\:11)$\\
$\equiv 3$

\section*{Question 2}
To find gcd(5n + 3, 7n + 4) we will use The Euclidean Algorithm.\\\\
$7n+4 = 1\cdot(5n+3)+(2n+1)$\\
$5n+3 = 2\cdot(2n+1)+(n+1)$\\
$2n+1 = 1\cdot(n+1)+n$\\
$n+1 = 1\cdot(n)+1$\\\\
Then gcd(5n + 3, 7n + 4) = 1
\newpage
\section*{Question 3}
We know that x is a prime number and, $m^2=n^2+k\cdot x$ equation holds for m,n,k integer values.\\
Then, we can write this equation as,\\
$m^2-n^2=k\cdot x$\\
$(m+n)\cdot(m-n)=k\cdot x$\\
For this equality to be hold, x should be the prime factor of either (m-n) or (m+n).\\
So this means $x|(m+n)\:or\:x|(m-n)$

\section*{Question 4}
Let P(n) be the proposition that the sum of all the integers that \\\\
P(1) is true, because 1 = $\frac{1\cdot(3\cdot1-1)}{2}$. 
For the inductive part, we assume that P(k) holds for an arbitrary positive integer k.\\
Namely, we assume that,\\
$1+4+7+...+(3\cdot k-2) = \frac{k\cdot(3\cdot k-1)}{2}$\\
Under this assumption, we should show that P(k+1) is true.\\
Namely,\\
$1+4+7+...+(3\cdot k-2)+(3\cdot k+1) = \frac{(k+1)\cdot(3\cdot(k+1)-1)}{2} = \frac{(k+1)\cdot(3\cdot k+2)}{2}$\\
is also true.\\
We add both sides $(3\cdot k+1)$.\\
$1+4+7+...+(3\cdot k-2)+(3\cdot k+1) =\frac{k\cdot(3\cdot k-1)}{2}+(3\cdot k+1)$\\
$1+4+7+...+(3\cdot k-2)+(3\cdot k+1) =\frac{k\cdot(3\cdot k-1)+2\cdot(3\cdot k+1)}{2}$\\
$1+4+7+...+(3\cdot k-2)+(3\cdot k+1) =\frac{(3\cdot k+2)\cdot(k+1)}{2}$\\
This last equation shows that P(k+1) is true under the assumtion that P(k) is true. This completes the inductive step.\\\\
In other words, we have proven $1+4+7+...+(3\cdot n-2) = \frac{n\cdot(3\cdot n-1)}{2}$

\end{document}

