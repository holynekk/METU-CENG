\documentclass[11pt]{article}
\usepackage[utf8]{inputenc}
\usepackage{float}
\usepackage{amsmath}
\usepackage{amssymb}

\usepackage[hmargin=3cm,vmargin=6.0cm]{geometry}
%\topmargin=0cm
\topmargin=-2cm
\addtolength{\textheight}{6.5cm}
\addtolength{\textwidth}{2.0cm}
%\setlength{\leftmargin}{-5cm}
\setlength{\oddsidemargin}{0.0cm}
\setlength{\evensidemargin}{0.0cm}

% symbol commands for the curious
\newcommand{\setZp}{\mathbb{Z}^+}
\newcommand{\setR}{\mathbb{R}}
\newcommand{\calT}{\mathcal{T}}

\begin{document}

\section*{Student Information } 
%Write your full name and id number between the colon and newline
%Put one empty space character after colon and before newline
Full Name :  Mert Kaan YILMAZ\\
Id Number :  2381093\\

% Write your answers below the section tags
\section*{Answer 1}
Picking a start from 10 distinct stars: $\binom{10}{1}$ = 10\\
Picking two habitable planets from 20 habitable planets: $\binom{20}{2}$\\
Picking eight non-habitable planets from 80 habitable planets: $\binom{80}{8}$\\
Then from product rule, choosing all the possibilities for selecting: $\binom{10}{1}\cdot\binom{20}{2}\cdot\binom{80}{8}$\\
Since the order of the planets matter, we should calculate all the permutations of the chosen planets: 10!\\\\
Therefore, the answer is: $\binom{10}{1}\cdot\binom{20}{2}\cdot\binom{80}{8}\cdot10! = 199860952\cdot10^{12}$


\section*{Answer 2}
Reorder the equation:
$a_n-2a_{n-1}-15a_{n-2}+36a_{n-3} = 2^n$\\
Let $a_n = r^3, a_{n-1} = r^2, a_{n-2} = r, a_{n-3} = 1$\\
The characteristic equation: $r^3-2r^2-15r+36 = 0 = (r+4)(r-3)^2$\\
So $r_1 = -4, r_{2,3} = 3$ (multiplicity of 2)\\\\
The solution of the homogeneous recurrence relation:\\
$a_n^{(h)} = A(-4)^n + (Bn+C)3^n$\\
Particular solution:\\
F(n) = $2^n$\\
Since 2 is not a root of the characteristic equation of the associated linear homogeneous recurrence relation, there is a particular solution of the form\\
$a_n^{(p)} = p_0(2)^n$ and this should satisfy the recurrence relation.\\
$a_n-2a_{n-1}-15a_{n-2}+36a_{n-3} = 2^n$\\
$ p_0(2)^n - 2p_0(2)^{n-1} - 15p_0(2)^{n-2} + 36p_0(2)^{n-3} = 2^n$ \:\:\:Divide both sides with $2^{n-3}$\\
$8p_0 - 8p_0 - 30p_0 + 36p_0 = 8 \:\:\rightarrow\:\: p_0 = 3/4$\\
Then $a_n^{(p)} = (3/4)(2)^n$\\\\
Therefore, general solution for the recurrence relation is:\\
$a_n = a_n^{(h)} + a_n^{(p)}n = A(-4)^n + (Bn+C)3^n + (3/4)(2)^n$

\newpage

\section*{Answer 3}
$a_1 = 5$\\
There are two mutual exclusive sets of valid activation codes\\
a. X $\rightarrow |$a valid code with length n-1$|$ + even digit\\
b. Y $\rightarrow |$an invalid code with length n-1$|$ + odd digit\\
$|X| = a_{n-1}\cdot5$\\
$|Y| = (10^{n-1}-a_{n-1})\cdot5$\\
Then,\\
$a_n = |X|+|Y|$\\
$a_n = 5\cdot a_{n-1} + 5\cdot (10^{n-1}-a_{n-1})$\\
$a_n = 5\cdot10^{n-1}$\\


\section*{Answer 4}
The equation can be written as: $a_k - 3\cdot a_{k-1}+3\cdot a_{k-2}-a_{k-3} = 0$\\
Let $f_{(x)}$ denote the generating function for the sequence $a_k$; that is, $f_{(x)}=\sum_{k\geq0}^na_k\cdot x^k$\\
Taking the first equation, multiply each term by $x^k$ and sum each term over all positive k$\geq$3. \\
$\sum_{k\geq3}^na_k\cdot x^k - 3\cdot\sum_{k\geq3}^na_{k-1}\cdot x^k + 3\cdot\sum_{k\geq3}^na_{k-2}\cdot x^k - \sum_{k\geq3}^na_{k-3}\cdot x^k = 0$\\\\
Manipulate each term so that we can write them as expressions in terms of the generating function $f_{(x)}$ and known series representations. \\\\
$\sum_{k\geq3}^na_k\cdot x^k = \sum_{k\geq3}^na_k\cdot x^k + a_0 + a_1x + a_2x^2 - a_0 - a_1x - a_2x^2 = \sum_{k\geq0}^na_k\cdot x^k - a_0 - a_1 - a_2x^2 \\= f(x) - 1 - 3x - 6x^2$\\\\
$\sum_{k\geq3}^na_{k-1}\cdot x^{k} = x\cdot\sum_{k\geq3}^na_{k-1}\cdot x^{k-1} = x\cdot\sum_{k\geq2}^na_k\cdot x^k = x\cdot(\sum_{k\geq0}^na_k\cdot x^k - a_1x - a_0) = x(f(x) - 3x - 1)$\\\\
$\sum_{k\geq3}^na_{k-2}\cdot x^{k} = x^2\cdot\sum_{k\geq3}^na_{k-2}\cdot x^{k-2} = x^2\cdot\sum_{k\geq1}^na_k\cdot x^k = x\cdot(\sum_{k\geq0}^na_k\cdot x^k - a_0) = x^2(f(x) - 1)$\\\\
$\sum_{k\geq3}^na_{k-3}\cdot x^{k} = x^3\cdot\sum_{k\geq3}^na_{k-3}\cdot x^{k-3} = x^3\cdot\sum_{k\geq0}^na_k\cdot x^k = x^3f(x)$\\\\
Then, our equation will be:\\
$f(x)-1-3x-6x^2-3xf(x)+9x^2+3x+3x^2f(x)-3x^2-x^3f(x) = 0$\\
$f(x)-3xf(x)+3x^2f(x)-x^3f(x) = 1$\\
$f(x)(1-x)^3 = 1$\\
$f(x) = \frac{1}{(1-x)^3}$\\\\
From book, page 542: $\frac{1}{(1-x)^n} = \sum_{k=0}^nC(n+k-1,k)\cdot x^{k} and a_k = C(n+k-1,k)$\\
Therefore, for n = 3\\ $a_k = C(2+k, k)$
\newpage
\section*{Answer 5}
\paragraph{a.}
For the given relation to be an equivalence class, it should be reflexive, symmetric, and transitive.\\
$\textbf{Reflexivity}$\\
Consider (a,b)R(a,b), (a,b) on $\setZp$x $\setZp$\\
Apply given relation condition: a+b = b+a\\
Since this is true for all (a,b), R is reflexive\\\\
$\textbf{Symmetry}$\\
Given (a,b)R(c,d) such that a+d=b+c\\
Consider (c,d)R(a,b) on $\setZp$x $\setZp$\\
Given relation condition says that: c+b=d+a.\\
Because this satisfies the given relation condition, R is symmetric.\\\\
$\textbf{Transitivity}$\\
Let (a,b)R(c,d) and (c,d)R(m,n), and all those pairs are in the set of $\setZp x\setZp$\\
Apply given relation condition:\\
c+n=d+m (*)\\
a+d=b+c (can also be written as a-c=b-d(**))\\
Add (*) and (**) side by side,
a-c+c+n = b-d+d+m\\
a+n=b+m $\rightarrow$ (a,b)R(m,n) also satisfies the condition.\\
Therefore R is transitive.\\\\
Since we showed that R is reflexive, symmetric, and transitive, R is an equivalence relation.\\
\paragraph{b.}
This class consist of all real pairs (a,b) satisfying (a,b)R(1,2).\\
Apply the given relation condition: 1+b = 2+a $\rightarrow$ b = a+1\\
Therefore, the equivalence class of (1,2): [(1,2)] = \{(a,a+1)$|$a$\in\setZp$\}

\end{document}
