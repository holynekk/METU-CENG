\documentclass[12pt]{article}
\usepackage[utf8]{inputenc}
\usepackage{float}
\usepackage{amsmath}


\usepackage[hmargin=3cm,vmargin=6.0cm]{geometry}
\topmargin=-2cm
\addtolength{\textheight}{6.5cm}
\addtolength{\textwidth}{2.0cm}
\setlength{\oddsidemargin}{0.0cm}
\setlength{\evensidemargin}{0.0cm}
\usepackage{indentfirst}
\usepackage{amsfonts}

\begin{document}

\section*{Student Information}

Name : Mert Kaan YILMAZ\\

ID : 2381093\\


\section*{Answer 1}
\subsection*{a)}\noindent 
Expected value for blue dice:\\
2$\cdot$P(2) + 3$\cdot$P(3) + 4$\cdot$P(4)\\
= 2$\cdot\frac{4}{6}$ + 3$\cdot\frac{1}{6}$ + 4$\cdot\frac{1}{6}$ = $\frac{15}{6}$ = 2.5\\\\
Expected value for yellow dice:\\
1$\cdot$P(1) + 2$\cdot$P(2) + 3$\cdot$P(3)\\
= 1$\cdot\frac{2}{6}$ + 2$\cdot\frac{2}{6}$ + 3$\cdot\frac{2}{6}$ = $\frac{12}{6}$ = 2\\\\
Expected value for red dice:\\
1$\cdot$P(1) + 2$\cdot$P(2) + 3$\cdot$P(3) + 5$\cdot$P(5)\\
= 1$\cdot\frac{2}{8}$ + 2$\cdot\frac{2}{8}$ + 3$\cdot\frac{3}{8}$ + 5$\cdot\frac{1}{8}$ = $\frac{20}{8}$ = 2.5\\
\subsection*{b)}\noindent 
By using expected values that are calculated in part a, Let's say expected values of blue, yellow, and red dice are $E_B, E_Y, E_R$ respectively. Then,\\\\
Expected values of 2 red and 1 yellow dice is: 2$\cdot E_R$ + $E_Y$ = 2$\cdot$2.5 + 2 = 7\\
Expected values of 2 yellow and 1 blue dice is: 2$\cdot E_Y$ + $E_B$ = 2$\cdot$2 + 2.5 = 6.5\\\\
Since expected value of first option is higher than the second one, I would pick "2 red and 1 yellow" (first option) to maximize total value.

\subsection*{c)}\noindent 
Basically, we say $E_B$ = 4. Then, expected value of second option will be $E_Y$+4 .
Since expected value of the second option will be 8, which is higher than the first one, choosing second option would give me the maximum total number.

\subsection*{d)}\noindent 
R, B, and Y are events of choosing red, blue and yellow dice respectively.\\
We know that, we got outcome of 3 from a dice. Also, our sample space is "getting 3 from blue OR red OR yellow dice". Then, getting a 3 from a red dice probability can be calculated as:\\
$\frac{P(R)\cdot P(3\vert R)}{P(B)\cdot P(3\vert B)+P(Y)\cdot P(3\vert Y)+P(R)\cdot P(3\vert R)}$ = $\frac{\frac{1}{3}\cdot \frac{3}{8}}{\frac{1}{3}\cdot \frac{1}{6}+\frac{1}{3}\cdot \frac{2}{6}+\frac{1}{3}\cdot \frac{3}{8}}$ = $\frac{3}{7}$

\subsection*{e)}\noindent 
Possibilities are:\\
(Red: 3 , Yellow: 3) and (Red: 5 , Yellow: 1)\\
$P_A(B)$ is the possibility of B outcome for dice A.\\
Possibility of the total value 6: $P_R(3)\cdot P_Y(3)$ + $P_R(5)\cdot P_Y(1)$=$\frac{3}{8}\cdot\frac{2}{6}+\frac{1}{8}\cdot\frac{2}{6}=\frac{1}{6}$\\


\section*{Answer 2}

\subsection*{a)}\noindent 
There are no electric outages in Ankara and two electric outages in Istanbul: a = 0 and i = 2. Therefore, \\
P(A=0, I=2) = 0.17

\subsection*{b)}\noindent 
There are two electric outages in Ankara and no electric outages in Istanbul: a = 2, i = 0.\\
Since 'a' value can take either 0 or 1, there is no possibility for 'a' to take 2. Therefore P(A=2, I=0) = 0

\subsection*{c)}\noindent 
There are two electric outages in total: a + i = 2\\
Possible values of a and i = [ (a=0, i=2), (a=1, i=1) ]\\
Therefore the result is summation of that two outcomes:\\
P(A=0, I=2) + P(A=1, I=1) = 0.17 + 0.11 = 0.28

\subsection*{d)}\noindent 
Single electric outage in Ankara: Since a is 0 or 1, a can only take value 1 and i can be anything(0, 1, 2, 3). Therefore,\\
the probability is: P(A=1, I=0) + P(A=1, I=1) + P(A=1, I=2) +P(A=1, I=3) = 0.12 + 0.11 + + 0.22 + 0.15 = 0.60

\subsection*{e)}\noindent 
All possible values of total number of outages is between 0 and 4(both are included). Let's say Y = A + I, then,\\
$P_Y$(0) = P\{Y = 0\} = P(A=0, I=0) = 0.08\\
$P_Y$(1) = P\{Y = 1\} = P(A=0, I=1) + P(A=1, I=0) = 0.13 + 0.12 = 0.25\\
$P_Y$(2) = P\{Y = 2\} = P(A=0, I=2) + P(A=1, I=1) = 0.17 + 0.11 = 0.28\\
$P_Y$(3) = P\{Y = 3\} = P(A=0, I=3) + P(A=1, I=2) = 0.02 + 0.22 = 0.24\\
$P_Y$(4) = P\{Y = 4\} = P(A=1, I=3) = 0.15

\subsection*{f)}\noindent 
They are not independent and to show that, we need to find an example which holds P(A$\cap$I)$\neq$P(A)$\cdot$P(I).\\
For example, let number of outages as A = 1 and I = 0. Then P(A=1 $\cap$ I=1) = 0.11 where P(A=1) = $\frac{1}{2}$ and P(I=1) = $\frac{1}{4}$ since there are 2 options for 'a' and 4 options for 'i'.\\
Therefore, P(A=1)$\cdot$P(I=1) = $\frac{1}{2}\cdot\frac{1}{4} = \frac{1}{8} = 12.5$.\\\\
Because P(A=1$\cap$I=1)$\neq$P(A=1)$\cdot$P(I=1), we can say that electric outages in Ankara and Istanbul are not independent.

\end{document}
